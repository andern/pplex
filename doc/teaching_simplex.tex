\documentclass[ukenglish]{nik}
\usepackage{mathptm}
\usepackage{hyperref}
\usepackage{amsfonts,amsmath,amssymb,amstext}
\usepackage{latexsym}
\usepackage{url}

\newcommand{\va}{\mathit{variable}}
\newcommand{\Fa}{\mathrm{F}}
\newcommand{\TEXT}{{\mathrm{TEXT}}}
\newcommand{\Nat}{\mathbb{N}}
\newcommand{\Int}{\mathbb{Z}}
\newcommand{\Rat}{\mathbb{Q}}
\newcommand{\Rea}{\mathbb{R}}
\newcommand{\cS}{\mathcal{S}}
\newcommand{\cD}{\mathcal{D}}
\newcommand{\cT}{\mathcal{T}}
\newcommand{\cL}{\mathcal{L}}
\newcommand{\cB}{\mathcal{B}}
\newcommand{\cF}{\mathcal{F}}
\newcommand{\cN}{\mathcal{N}}
\newcommand{\cP}{\mathcal{P}}
\newcommand{\cA}{\mathcal{A}}
\newcommand{\ra}{\rightarrow}
\newcommand{\Ra}{\Rightarrow}
\newcommand{\set}[1]{\{#1\}}
\newcommand{\seg}[2]{[#1.\,.#2]}
\newcommand{\pair}[2]{\langle #1,#2\rangle}
\newcommand{\der}[2]{\frac{\partial#1}{\partial#2}}
\newcommand{\plusD}[1]{#1+\Delta{#1}}
\newcommand{\dD}[1]{\Delta{#1}}
\newcommand{\range}{\mathit{range}}
\newcommand{\oa}{\overline{a}}
\newcommand{\ob}{\overline{b}}
\newcommand{\oc}{\overline{c}}
\newcommand{\oz}{\overline{\zeta}}
\newcommand{\ea}{\epsilon_1}
\newcommand{\eb}{\epsilon_2}
\newcommand{\ec}{\epsilon_3}
\newcommand{\ba}{\frac{100}{2}b_1}
\newcommand{\bb}{\frac{10}{2}b_2}
\newcommand{\bc}{\frac{1}{2}b_3}
\newcommand{\binomsq}[2]{\left[\begin{array}{c}#1\\#2\end{array}\right]}



\begin{document}

\title{Teaching the Simplex Method}

\author{
Joanna Bauer\thanks{University of Bergen, Department of Informatics, P.O.Box 7803, N-5020 Bergen, Norway}
\and
Marc Bezem$^*$
\and
Andreas Halle$^*$}
\maketitle

\begin{abstract}
We present software (\url{https://github.com/andern/lpped})
for the classroom presentation of the simplex method in linear programming.
\end{abstract}

\section{Motivation}
In most courses on linear programming the need arises for demonstrating the simplex method step-by-step 
\emph{without being distracted by detailed calculations in elementary linear algebra}. Let us start by
introducing the simplex method and at the same time presenting the teaching issue at hand. 

 \begin{itemize}
    \item Maximize $5 x_1 + 4 x_2 + 3 x_3$ subject to
    \begin{itemize}
      \item $2 x_1 + 3 x_2 + x_3 \leq 5$
      \item $4 x_1 + x_2 + 2 x_3 \leq 11$
      \item $3 x_1 + 4 x_2 + 2 x_3 \leq 8$
      \item all variables $\geq 0$
    \end{itemize}
    \item Introduce slack variables in the above \emph{standard form}
    \item Maximize $\zeta$ subject to (with all variables $\geq 0$):
\[    
    \begin{array}{lcrcrcrcr}
      \zeta&=&   & & 5 x_1 &+& 4 x_2 &+& 3 x_3 \\\hline
      w_1  &=& 5 &-& 2 x_1 &-& 3 x_2 &-&   x_3 \\
      w_2  &=& 11&-& 4 x_1 &-&   x_2 &-& 2 x_3 \\
      w_3  &=& 8 &-& 3 x_1 &-& 4 x_2 &-& 2 x_3 
    \end{array}
\]
  \end{itemize}

Further desiderata:

1. Open source

2. Up-to-date

3. Free and independent of licences

4. Supporting most standards for LP problems

\section{Examples}
1. primal feasible

2. dual feasible but not primal

3. primal and dual infeasible

4. unbounded

5. cyclic

\section{Features}
Short overview of the software, dicussion of design decisions.

\section{Future extensions}




\bibliographystyle{abbrv}
\begin{thebibliography}{99}
\bibitem{lpped}
\url{https://github.com/andern/lpped}
%\bibitem{WalkerAMM}
%Peter Walker, A Lemma on Divisibility. \emph{American Mathematical Monthly}
%{\bf 115}(4):338.
\end{thebibliography}



\end{document}
