\documentclass[ukenglish]{nik}
\usepackage{mathptm}
\usepackage{hyperref}
\usepackage{amsfonts,amsmath,amssymb,amstext}
\usepackage{latexsym}
\usepackage{url}

\newcommand{\va}{\mathit{variable}}
\newcommand{\Fa}{\mathrm{F}}
\newcommand{\TEXT}{{\mathrm{TEXT}}}
\newcommand{\Nat}{\mathbb{N}}
\newcommand{\Int}{\mathbb{Z}}
\newcommand{\Rat}{\mathbb{Q}}
\newcommand{\Rea}{\mathbb{R}}
\newcommand{\cS}{\mathcal{S}}
\newcommand{\cD}{\mathcal{D}}
\newcommand{\cT}{\mathcal{T}}
\newcommand{\cL}{\mathcal{L}}
\newcommand{\cB}{\mathcal{B}}
\newcommand{\cF}{\mathcal{F}}
\newcommand{\cN}{\mathcal{N}}
\newcommand{\cP}{\mathcal{P}}
\newcommand{\cA}{\mathcal{A}}
\newcommand{\ra}{\rightarrow}
\newcommand{\Ra}{\Rightarrow}
\newcommand{\set}[1]{\{#1\}}
\newcommand{\seg}[2]{[#1.\,.#2]}
\newcommand{\pair}[2]{\langle #1,#2\rangle}
\newcommand{\der}[2]{\frac{\partial#1}{\partial#2}}
\newcommand{\plusD}[1]{#1+\Delta{#1}}
\newcommand{\dD}[1]{\Delta{#1}}
\newcommand{\range}{\mathit{range}}
\newcommand{\oa}{\overline{a}}
\newcommand{\ob}{\overline{b}}
\newcommand{\oc}{\overline{c}}
\newcommand{\oz}{\overline{\zeta}}
\newcommand{\ea}{\epsilon_1}
\newcommand{\eb}{\epsilon_2}
\newcommand{\ec}{\epsilon_3}
\newcommand{\ba}{\frac{100}{2}b_1}
\newcommand{\bb}{\frac{10}{2}b_2}
\newcommand{\bc}{\frac{1}{2}b_3}
\newcommand{\binomsq}[2]{\left[\begin{array}{c}#1\\#2\end{array}\right]}



\begin{document}

\title{Teaching the Simplex Method}

\author{
Joanna Bauer\thanks{University of Bergen, Department of Informatics, P.O.Box 7803, N-5020 Bergen, Norway}
\and
Marc Bezem$^*$
\and
Andreas Halle$^*$}
\maketitle

\begin{abstract}
We present software (\url{https://github.com/andern/lpped})
for the classroom presentation of the simplex method in linear programming.
\end{abstract}

\section{Motivation}
Linear Programming (LP) is the subfield of mathematical optimization where
the aim is to maximize a linear function under linear constraints. Such a problem
is called an LP problem. The simplex method is a family of algorithms for solving LP problems. 
Although there exist other algorithms for solving LP problems,
the simplex method is by far the one that is most used in practice.
It is also listed it as one of the top 10 algorithms of the twentieth century in
\cite{CiSaE2000}. Given the many applications of LP in economics and engineering and
management, it is deemed important to teach the simplex method to a wide range of
students. This range certainly includes students in applied mathematics and informatics,
but extends considerably beyond these groups. The fact that the simplex method is
taught to students with a weaker background in mathematics and algorithmics than
the abovementioned two groups should have important consequences for the way
in which we teach the simplex method.

In most courses on linear programming the need arises for demonstrating the simplex method step-by-step 
\emph{without being distracted by detailed calculations in elementary linear algebra}. Let us start by
introducing the simplex method by a simple example:
%and at the same time present some teaching issues at hand. 
\[
    \begin{array}{llcrcrcrcrcrcr}
    \text{Maximize}  &    x &+&    y \\
    \text{subject to}&  2 x &+&    y &\leq&   6\\
    \text{and}&         7 x &+& 13 y &\leq&  40\\
    \text{and}       &    x &,&    y &\geq&0    &
    \end{array}
\]
where the variables $x$ and $y$ range over the real numbers.
A first observation is that there exist feasible solutions,
since the polygon given by the four (yes, four!) inequalities contains,
for example, the origin. A first systematic step is to give these inequalities
all the same format $\mathit{variable}\geq 0$. For this purpose we introduce
two new variables, $u$ and $v$, measuring to what extent the first
two inequalities are satisfied. These new variables are commonly called
\emph{slack variables}. At the same time we introduce an extra variable
for the function $x+y$, the so-called \emph{objective function},
or \emph{objective} for short. Thus our LP problem is brought in
\emph{standard form}:
\[    
    \begin{array}{lcrcrcrcr}
      \zeta&=&   & &     x &+&     y & &  \\\hline
      u    &=& 6 &-&   2 x &-&     y & &  \\
      v    &=& 40&-&   7 x &-&  13 y & &  \\
      x,y,u,v  &\geq&0     & &  & &  & &    
    \end{array}
\]
Note that each edge of the polygon corresponds to a variable
being zero on that edge. Consequently,
vertices of the polygon are all given by a pair of
two distinct variables being zero. On the other hand, some of these
pairs take their values zero outside the polygon, f.e.\ , $x=v=0$.
Since the inequalities $x,y,u,v \geq 0$ are part of the standard, 
they are often omitted. 

Geometric considerations lead to the observation that the
maximal value of the objective under the constraints (linear
equalities and inequalities of the form 
$\mathit{variable}\geq 0$ is attained in a vertex of the
polygon. The idea of the standard form is
that the variables on the right (the \emph{basic} variables) are zero,
giving the variables on the left feasible, that is non-negative, values.

In the example, we start in the origin $x=y=0$ and get
value 0 for the objective $\zeta = x + y$.
Clearly, there is room for improving here, since actually 
the value 0 is the \emph{minimal} value of the objective
under the given constraints. How can one improve on this
value? The answer is: increase $x$ or $y$ or both.
In the simplex method one chooses one basic variable and
tries to increase it as much as possible, while keeping the
other basic variables.
Let's try to increase $x$ keeping $y=0$. How much can $x$ increase?
We see that $0 \leq u = 6 -2x-y$ allows us to increase $x$ to $3$
and that the other linear equality allows even more. Since both
$u\geq 0$ and $v\geq 0$ must be respected, we increase $x$ to $3$,
thereby getting $u=0$ as a consequence. Moreover the objective $x+y$ becomes 3.

With $x$ increased to 3, $x$ doesn't qualify as a basic variable anymore.
Fortunately, the variable $u$ has become 0. Exchanging the roles of
$x$ and $u$ as basic and non-basic variable, respectively, restores 
the invariant of the standard form. The first equation $u = 6 -2x-y$ makes it
possible to express $x$ in $u,y$ such that $x$ can be eliminated from the
right-hand side of the standard form. Technically this is done by standard
row operations in linear algebra. The new standard form becomes:

\[    
    \begin{array}{lcrcrcrcr}
      \zeta&=& 3 &-& \frac{1}{2} u &+& \frac{1}{2} y & &  \\\hline
      u    &=& 6 &-&   2 x &-&     y & &  \\
      v    &=& 40&-&   7 x &-&  13 y & &  \\

    \end{array}
\]
 



After being brought in standard form, the next step in the simplex
method is picking a 


Further desiderata:

1. Open source

2. Up-to-date

3. Free and independent of licences

4. Supporting most standards for LP problems

\section{Examples}
1. primal feasible

2. dual feasible but not primal

3. primal and dual infeasible

4. unbounded

5. cyclic

\section{Features}
Short overview of the software, dicussion of design decisions.

\section{Future extensions}




\bibliographystyle{abbrv}
\begin{thebibliography}{99}
\bibitem{CiSaE2000} Computing in Science and Engineering, volume 2, no. 1, 2000.

\bibitem{lpped}
\url{https://github.com/andern/lpped}
%\bibitem{WalkerAMM}
%Peter Walker, A Lemma on Divisibility. \emph{American Mathematical Monthly}
%{\bf 115}(4):338.
\end{thebibliography}



\end{document}
