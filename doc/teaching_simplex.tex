\documentclass[ukenglish]{nik}
\usepackage{mathptm}
\usepackage{hyperref}
\usepackage{amsfonts,amsmath,amssymb,amstext}
\usepackage{latexsym}
\usepackage{url}

\newcommand{\va}{\mathit{variable}}
\newcommand{\Fa}{\mathrm{F}}
\newcommand{\TEXT}{{\mathrm{TEXT}}}
\newcommand{\Nat}{\mathbb{N}}
\newcommand{\Int}{\mathbb{Z}}
\newcommand{\Rat}{\mathbb{Q}}
\newcommand{\Rea}{\mathbb{R}}
\newcommand{\cS}{\mathcal{S}}
\newcommand{\cD}{\mathcal{D}}
\newcommand{\cT}{\mathcal{T}}
\newcommand{\cL}{\mathcal{L}}
\newcommand{\cB}{\mathcal{B}}
\newcommand{\cF}{\mathcal{F}}
\newcommand{\cN}{\mathcal{N}}
\newcommand{\cP}{\mathcal{P}}
\newcommand{\cA}{\mathcal{A}}
\newcommand{\ra}{\rightarrow}
\newcommand{\Ra}{\Rightarrow}
\newcommand{\set}[1]{\{#1\}}
\newcommand{\seg}[2]{[#1.\,.#2]}
\newcommand{\pair}[2]{\langle #1,#2\rangle}
\newcommand{\der}[2]{\frac{\partial#1}{\partial#2}}
\newcommand{\plusD}[1]{#1+\Delta{#1}}
\newcommand{\dD}[1]{\Delta{#1}}
\newcommand{\range}{\mathit{range}}
\newcommand{\oa}{\overline{a}}
\newcommand{\ob}{\overline{b}}
\newcommand{\oc}{\overline{c}}
\newcommand{\oz}{\overline{\zeta}}
\newcommand{\ea}{\epsilon_1}
\newcommand{\eb}{\epsilon_2}
\newcommand{\ec}{\epsilon_3}
\newcommand{\ba}{\frac{100}{2}b_1}
\newcommand{\bb}{\frac{10}{2}b_2}
\newcommand{\bc}{\frac{1}{2}b_3}
\newcommand{\binomsq}[2]{\left[\begin{array}{c}#1\\#2\end{array}\right]}



\begin{document}

\title{Teaching the Simplex Method}

\author{
Joanna Bauer\thanks{University of Bergen, Department of Informatics, P.O.Box 7803, N-5020 Bergen, Norway}
\and
Marc Bezem$^*$
\and
Andreas Halle$^*$}
\maketitle

\begin{abstract}
We present software (\url{https://github.com/andern/lpped})
for the classroom presentation of the simplex method in linear programming.
\end{abstract}

\section{Motivation}
Linear Programming (LP) is the subfield of mathematical optimization where
the aim is to maximize a linear function under linear constraints. Such a problem
is called an LP problem. The simplex method is an algorithm for solving LP problems. 
Although there exist other algorithms for solving LP problems,
the simplex method is by far the one that is most used in practice.
It is also listed it as one of the top 10 algorithms of the twentieth century in
\cite{CiSaE2000}. Given the many applications of LP in economics and engineering and
management, it is deemed important to teach the simplex method to a wide range of
students. This range certainly includes students in applied mathematics and informatics,
but extends considerably beyond these groups. The fact that the simplex method is
taught to students with a weaker background in mathematics and algorithmics than
the abovementioned two groups should have important consequences for the way
in which we teach the simplex method.

In most courses on linear programming the need arises for demonstrating the simplex method step-by-step 
\emph{without being distracted by detailed calculations in elementary linear algebra}. Let us start by
introducing the simplex method by a simpe example.

and at the same time presenting the teaching issue at hand. 

 \begin{itemize}
    \item Maximize $5 x_1 + 4 x_2 + 3 x_3$ subject to
    \begin{itemize}
      \item $2 x_1 + 3 x_2 + x_3 \leq 5$
      \item $4 x_1 + x_2 + 2 x_3 \leq 11$
      \item $3 x_1 + 4 x_2 + 2 x_3 \leq 8$
      \item all variables $\geq 0$
    \end{itemize}
    \item Introduce slack variables in the above \emph{standard form}
    \item Maximize $\zeta$ subject to (with all variables $\geq 0$):
\[    
    \begin{array}{lcrcrcrcr}
      \zeta&=&   & & 5 x_1 &+& 4 x_2 &+& 3 x_3 \\\hline
      w_1  &=& 5 &-& 2 x_1 &-& 3 x_2 &-&   x_3 \\
      w_2  &=& 11&-& 4 x_1 &-&   x_2 &-& 2 x_3 \\
      w_3  &=& 8 &-& 3 x_1 &-& 4 x_2 &-& 2 x_3 
    \end{array}
\]
  \end{itemize}

Further desiderata:

1. Open source

2. Up-to-date

3. Free and independent of licences

4. Supporting most standards for LP problems

\section{Examples}
1. primal feasible

2. dual feasible but not primal

3. primal and dual infeasible

4. unbounded

5. cyclic

\section{Features}
Short overview of the software, dicussion of design decisions.

\section{Future extensions}




\bibliographystyle{abbrv}
\begin{thebibliography}{99}
\bibitem{CiSaE2000} Computing in Science and Engineering, volume 2, no. 1, 2000.

\bibitem{lpped}
\url{https://github.com/andern/lpped}
%\bibitem{WalkerAMM}
%Peter Walker, A Lemma on Divisibility. \emph{American Mathematical Monthly}
%{\bf 115}(4):338.
\end{thebibliography}



\end{document}
