\documentclass[mathserif]{beamer}
%\documentclass[mathserif, handout]{beamer}

\usepackage{UiB_alternative_template_blue}

\usepackage{graphicx}
\usepackage[noline,noend,algoruled,linesnumbered]{algorithm2e}
\usepackage[retainorgcmds]{IEEEtrantools}
\usepackage{pause}
\usepackage[absolute,overlay]{textpos}
\usepackage{ulem}
\usepackage{hyperref}
\usepackage{amsfonts,amsmath,amssymb,amstext}
\usepackage{latexsym}
\usepackage{url}

\setbeamersize{text margin left=.5cm}

\begin{document}

\title{Teaching the Simplex Method}
\author{J.~Bauer, M.~Bezem, and A.~Halle} 

\begin{frame}
\titlepage
\end{frame}

\begin{frame}
{Geometry of Linear Programs}
  \begin{block}{Simple Example}
	 \nointerlineskip
	 \begin{IEEEeqnarray}{RL}
		\max 3x+2y & \nonumber \\
		\text{s.t. } 3x + y & \leq 13.5  \label{eq:ex1.1}\\
							x+3y & \leq 10.5 \label{eq:ex1.2}\\
							x,y  & \geq 0 \label{eq:ex1.3}
	 \end{IEEEeqnarray}
  \end{block}
	 \nointerlineskip
  \begin{block}{Geometric Observations}
	 \begin{itemize}[<+->]
		\item Inequalities (\ref{eq:ex1.1})--(\ref{eq:ex1.3}) correspond to half-planes\\ (half-spaces in higher dimensions)
		\item They define the feasible region
		\item The objective function corresponds to an ``improvement vector''
		\item If there is an optimal point, then there is an optimal corner point
	 \end{itemize}
  \end{block}
\end{frame}

\end{document}
